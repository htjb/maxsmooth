%% Generated by Sphinx.
\def\sphinxdocclass{report}
\documentclass[letterpaper,10pt,english]{sphinxmanual}
\ifdefined\pdfpxdimen
   \let\sphinxpxdimen\pdfpxdimen\else\newdimen\sphinxpxdimen
\fi \sphinxpxdimen=.75bp\relax

\PassOptionsToPackage{warn}{textcomp}
\usepackage[utf8]{inputenc}
\ifdefined\DeclareUnicodeCharacter
% support both utf8 and utf8x syntaxes
  \ifdefined\DeclareUnicodeCharacterAsOptional
    \def\sphinxDUC#1{\DeclareUnicodeCharacter{"#1}}
  \else
    \let\sphinxDUC\DeclareUnicodeCharacter
  \fi
  \sphinxDUC{00A0}{\nobreakspace}
  \sphinxDUC{2500}{\sphinxunichar{2500}}
  \sphinxDUC{2502}{\sphinxunichar{2502}}
  \sphinxDUC{2514}{\sphinxunichar{2514}}
  \sphinxDUC{251C}{\sphinxunichar{251C}}
  \sphinxDUC{2572}{\textbackslash}
\fi
\usepackage{cmap}
\usepackage[T1]{fontenc}
\usepackage{amsmath,amssymb,amstext}
\usepackage{babel}



\usepackage{times}
\expandafter\ifx\csname T@LGR\endcsname\relax
\else
% LGR was declared as font encoding
  \substitutefont{LGR}{\rmdefault}{cmr}
  \substitutefont{LGR}{\sfdefault}{cmss}
  \substitutefont{LGR}{\ttdefault}{cmtt}
\fi
\expandafter\ifx\csname T@X2\endcsname\relax
  \expandafter\ifx\csname T@T2A\endcsname\relax
  \else
  % T2A was declared as font encoding
    \substitutefont{T2A}{\rmdefault}{cmr}
    \substitutefont{T2A}{\sfdefault}{cmss}
    \substitutefont{T2A}{\ttdefault}{cmtt}
  \fi
\else
% X2 was declared as font encoding
  \substitutefont{X2}{\rmdefault}{cmr}
  \substitutefont{X2}{\sfdefault}{cmss}
  \substitutefont{X2}{\ttdefault}{cmtt}
\fi


\usepackage[Bjarne]{fncychap}
\usepackage{sphinx}

\fvset{fontsize=\small}
\usepackage{geometry}


% Include hyperref last.
\usepackage{hyperref}
% Fix anchor placement for figures with captions.
\usepackage{hypcap}% it must be loaded after hyperref.
% Set up styles of URL: it should be placed after hyperref.
\urlstyle{same}

\addto\captionsenglish{\renewcommand{\contentsname}{Contents:}}

\usepackage{sphinxmessages}
\setcounter{tocdepth}{1}



\title{maxsmooth Documentation}
\date{Jul 10, 2020}
\release{1.0.0}
\author{Harry Thomas Jones Bevins}
\newcommand{\sphinxlogo}{\vbox{}}
\renewcommand{\releasename}{Release}
\makeindex
\begin{document}

\pagestyle{empty}
\sphinxmaketitle
\pagestyle{plain}
\sphinxtableofcontents
\pagestyle{normal}
\phantomsection\label{\detokenize{source/index::doc}}



\chapter{Introduction}
\label{\detokenize{source/intro:introduction}}\begin{quote}\begin{description}
\item[{maxsmooth}] \leavevmode
Derivative Constrained Function Fitting

\item[{Author}] \leavevmode
Harry Thomas Jones Bevins

\item[{Version}] \leavevmode
1.0.0

\item[{Homepage}] \leavevmode
\sphinxurl{https://github.com/htjb/maxsmooth}

\end{description}\end{quote}


\section{Derivative Constrained Functions and \sphinxstyleliteralintitle{\sphinxupquote{maxsmooth}}}
\label{\detokenize{source/intro:derivative-constrained-functions-and-maxsmooth}}
\sphinxcode{\sphinxupquote{maxsmooth}} is an open source software for fitting derivative constrained
functions, DCFs such as Maximally Smooth Functions
, MSFs to data sets. MSFs are functions for which there are no zero
crossings in derivatives of order \(m \geq 2\) within the domain of interest.
They are designed to prevent the loss of
signals when fitting out dominant foregrounds and in some cases can be used to
highlight systematics left in the data. More generally for DCFs the minimum
constrained derivative order, m can take on any value or a set of
specific high order derivatives can be constrained.

You can read more about MSFs here ..

\sphinxcode{\sphinxupquote{maxsmooth}} uses quadratic programming implemented with \sphinxcode{\sphinxupquote{CVXOPT}} to fit
data subject to a linear constraint. The constraint on an MSF can be
summarized like so,
\begin{equation*}
\begin{split}\frac{d^m~y}{d~x^m}~\geq~0~~\mathrm{or}~~\frac{d^m~y}{d~x^m}~\leq~0.\end{split}
\end{equation*}
This constraint is itself not linear but \sphinxcode{\sphinxupquote{maxsmooth}} is designed to test the
constraint,
\begin{equation*}
\begin{split}\pm \frac{d^m~y}{d~x^m}~\leq~0\end{split}
\end{equation*}
where a positive sign in front of the \(m^{th}\) order derivative forces the derivative
to be negative for all x. For an \(N^{th}\) order polynomial \sphinxcode{\sphinxupquote{maxsmooth}} can test
every available sign combination but by default it implements a \textquotesingle{}sign\sphinxhyphen{}sampling\textquotesingle{}/\textquotesingle{}sign\sphinxhyphen{}flipping\textquotesingle{}
algorithm. This is detailed in the \sphinxcode{\sphinxupquote{maxsmooth}} paper (see citation) but is summarized
below.

The available sign combinations act as discrete parameter spaces all with
global minima and \sphinxcode{\sphinxupquote{maxsmooth}} is capable of finding the minimum of these global
minima by implementing a descent algorithm which is followed by a directional
exploration. The descent routine typically finds an approximate to the global
minimum and then the directional exploration is a complete search
of the sign combinations in the neighbourhood
of that minimum. The searched region is limited by factors
that encapsulate enough of the neighbourhood to confidently return the global minimum.

The sign sampling method is reliant on the problem being \textquotesingle{}well defined\textquotesingle{} but this
is not always the case and it is in these instances possible to run the code testing
every available sign combination on the constrained derivatives. For a definition of
a \textquotesingle{}well defined\textquotesingle{} problem and it\textquotesingle{}s counter part see the \sphinxcode{\sphinxupquote{maxsmooth}} paper.

\sphinxcode{\sphinxupquote{maxsmooth}} features a built in library of DCFs or
allows the user to define their own. The addition of possible inflection points
and zero crossings in higher order derivatives is also available to the user.
The software has been designed with these two
applications in mind and is a simple interface.


\section{Example Fit}
\label{\detokenize{source/intro:example-fit}}
Shown below is an example MSF fit performed with \sphinxcode{\sphinxupquote{maxsmooth}} to data that
follows a \({y = x^{-2.5}}\) power law with a randomly generated Gaussian
noise with a standard deviation 0.02. The top panel shows the data and the
bottom panel shows the residual
after subtraction of the MSF fit. The software using one of the built in DCF models
and fitting normalised data is shown to be capable of recovering the
random noise.

\noindent{\hspace*{\fill}\sphinxincludegraphics[width=400\sphinxpxdimen]{{README}.png}\hspace*{\fill}}


\section{Installation}
\label{\detokenize{source/intro:installation}}

\section{Licence and Citation}
\label{\detokenize{source/intro:licence-and-citation}}
The software is free to use on the MIT open source license. However if you use
the software for academic purposes we request that you cite the \sphinxcode{\sphinxupquote{maxsmooth}}
paper.

\begin{sphinxVerbatim}[commandchars=\\\{\}]
\PYG{n}{H}\PYG{o}{.} \PYG{n}{T}\PYG{o}{.} \PYG{n}{J}\PYG{o}{.} \PYG{n}{Bevins} \PYG{n}{et} \PYG{n}{al}\PYG{o}{.}\PYG{p}{,} \PYG{o+ow}{in} \PYG{n}{prep}\PYG{o}{.}
\end{sphinxVerbatim}


\section{Documentation}
\label{\detokenize{source/intro:documentation}}
The documentation can be compiled locally from the git repository and requires
\sphinxhref{https://pypi.org/project/Sphinx/}{sphinx} to be installed.
You can do this via:

\begin{sphinxVerbatim}[commandchars=\\\{\}]
\PYG{n}{cd} \PYG{n}{docs}\PYG{o}{/}
\PYG{n}{make} \PYG{n}{html}
\end{sphinxVerbatim}

or

\begin{sphinxVerbatim}[commandchars=\\\{\}]
\PYG{n}{cd} \PYG{n}{docs}\PYG{o}{/}
\PYG{n}{make} \PYG{n}{latexpdf}
\end{sphinxVerbatim}

The resultant docs can be found in the docs/\_build/html/ and docs/\_build/latex/
respectively.


\section{Requirements}
\label{\detokenize{source/intro:requirements}}
The code was written in Python 3.6 but should be backward compatible with
Python 2 although this has not been thoroughly tested.

To run the code you will need the following additional packages:
\begin{itemize}
\item {} 
\sphinxhref{https://pypi.org/project/matplotlib/}{matplotlib}

\item {} 
\sphinxhref{https://pypi.org/project/numpy/}{numpy}

\item {} 
\sphinxhref{https://pypi.org/project/cvxopt/}{CVXOPT}

\item {} 
\sphinxhref{https://pypi.org/project/scipy/}{scipy}

\item {} 
\sphinxhref{https://pypi.org/project/progressbar/}{progressbar}

\end{itemize}

To compile the documentation locally you will need:
\begin{itemize}
\item {} 
\sphinxhref{https://pypi.org/project/Sphinx/}{sphinx}

\item {} 
\sphinxhref{https://pypi.org/project/numpydoc/}{numpydoc}

\end{itemize}

To run the test suit you will need:
\begin{itemize}
\item {} 
\sphinxhref{https://pypi.org/project/pytest/}{pytest}

\end{itemize}


\chapter{Maxsmooth Example Codes}
\label{\detokenize{source/maxsmooth:maxsmooth-example-codes}}\label{\detokenize{source/maxsmooth::doc}}
This section is designed to introduce the user to the software and the form
in which it is run. It provides basic examples of data fitting with a built in
MSF model and a user defined model.

There are also examples of functions that can be used pre\sphinxhyphen{}fitting and post\sphinxhyphen{}fitting
for various purposes including; determination of the best DCF model from the
built in library for the problem being fitted, analysis of the \({\chi^2}\)
distribution as a function of the discrete sign spaces and analysis of the
parameter space surrounding the optimum results.


\section{Simple Example code}
\label{\detokenize{source/maxsmooth:simple-example-code}}
In order to run the \sphinxcode{\sphinxupquote{maxsmooth}} software using the built
in DCF models for a simple fit the user can follow the simple structure detailed here.

The user should begin by importing the \sphinxtitleref{smooth} class from \sphinxtitleref{maxsmooth.DCF}.

\begin{sphinxVerbatim}[commandchars=\\\{\}]
\PYG{k+kn}{from} \PYG{n+nn}{maxsmooth}\PYG{n+nn}{.}\PYG{n+nn}{DCF} \PYG{k+kn}{import} \PYG{n}{smooth}
\end{sphinxVerbatim}

The user should then import the data they wish to fit.

\begin{sphinxVerbatim}[commandchars=\\\{\}]
\PYG{k+kn}{import} \PYG{n+nn}{numpy} \PYG{k}{as} \PYG{n+nn}{np}

\PYG{n}{x} \PYG{o}{=} \PYG{n}{np}\PYG{o}{.}\PYG{n}{load}\PYG{p}{(}\PYG{l+s+s1}{\PYGZsq{}}\PYG{l+s+s1}{Data/x.npy}\PYG{l+s+s1}{\PYGZsq{}}\PYG{p}{)}
\PYG{n}{y} \PYG{o}{=} \PYG{n}{np}\PYG{o}{.}\PYG{n}{load}\PYG{p}{(}\PYG{l+s+s1}{\PYGZsq{}}\PYG{l+s+s1}{Data/y.npy}\PYG{l+s+s1}{\PYGZsq{}}\PYG{p}{)}
\end{sphinxVerbatim}

and define the polynomial orders they wish to fit.

\begin{sphinxVerbatim}[commandchars=\\\{\}]
N = [3, 4, 5, 6, 7, 8, 9, 10, 11]
for i in range(len(N)):
    `act on N[i]`
\end{sphinxVerbatim}

or for example,

\begin{sphinxVerbatim}[commandchars=\\\{\}]
\PYG{n}{N} \PYG{o}{=} \PYG{l+m+mi}{10}
\end{sphinxVerbatim}

\sphinxtitleref{smooth} can be called like so,

\begin{sphinxVerbatim}[commandchars=\\\{\}]
\PYG{n}{result} \PYG{o}{=} \PYG{n}{smooth}\PYG{p}{(}\PYG{n}{x}\PYG{p}{,} \PYG{n}{y}\PYG{p}{,} \PYG{n}{N}\PYG{p}{,} \PYG{o}{*}\PYG{o}{*}\PYG{n}{kwargs}\PYG{p}{)}
\end{sphinxVerbatim}

where the kwargs are detailed below. It\textquotesingle{}s resulting attributes can be accessed by writing
\sphinxcode{\sphinxupquote{result.attribute\_name}}. For example printing the outputs is done like
so,

\begin{sphinxVerbatim}[commandchars=\\\{\}]
\PYG{n+nb}{print}\PYG{p}{(}\PYG{l+s+s1}{\PYGZsq{}}\PYG{l+s+s1}{Objective Funtion Evaluations:}\PYG{l+s+se}{\PYGZbs{}n}\PYG{l+s+s1}{\PYGZsq{}}\PYG{p}{,} \PYG{n}{result}\PYG{o}{.}\PYG{n}{optimum\PYGZus{}chi}\PYG{p}{)}
\PYG{n+nb}{print}\PYG{p}{(}\PYG{l+s+s1}{\PYGZsq{}}\PYG{l+s+s1}{RMS:}\PYG{l+s+se}{\PYGZbs{}n}\PYG{l+s+s1}{\PYGZsq{}}\PYG{p}{,} \PYG{n}{result}\PYG{o}{.}\PYG{n}{rms}\PYG{p}{)}
\PYG{n+nb}{print}\PYG{p}{(}\PYG{l+s+s1}{\PYGZsq{}}\PYG{l+s+s1}{Parameters:}\PYG{l+s+se}{\PYGZbs{}n}\PYG{l+s+s1}{\PYGZsq{}}\PYG{p}{,} \PYG{n}{result}\PYG{o}{.}\PYG{n}{optimum\PYGZus{}params}\PYG{p}{)}
\PYG{n+nb}{print}\PYG{p}{(}\PYG{l+s+s1}{\PYGZsq{}}\PYG{l+s+s1}{Fitted y:}\PYG{l+s+se}{\PYGZbs{}n}\PYG{l+s+s1}{\PYGZsq{}}\PYG{p}{,} \PYG{n}{result}\PYG{o}{.}\PYG{n}{y\PYGZus{}fit}\PYG{p}{)}
\PYG{n+nb}{print}\PYG{p}{(}\PYG{l+s+s1}{\PYGZsq{}}\PYG{l+s+s1}{Sign Combinations:}\PYG{l+s+se}{\PYGZbs{}n}\PYG{l+s+s1}{\PYGZsq{}}\PYG{p}{,} \PYG{n}{result}\PYG{o}{.}\PYG{n}{optimum\PYGZus{}signs}\PYG{p}{)}
\PYG{n+nb}{print}\PYG{p}{(}\PYG{l+s+s1}{\PYGZsq{}}\PYG{l+s+s1}{Derivatives:}\PYG{l+s+se}{\PYGZbs{}n}\PYG{l+s+s1}{\PYGZsq{}}\PYG{p}{,} \PYG{n}{result}\PYG{o}{.}\PYG{n}{derivatives}\PYG{p}{)}
\end{sphinxVerbatim}


\section{New Basis Example}
\label{\detokenize{source/maxsmooth:new-basis-example}}
This example code illustrates how to define your own basis function for the
DCF model.
It implements a modified version of the built in normalized polynomial model
but the structure is the same for more elaborate models.

As always we need to import the data, define an order \({N}\)
and import the function fitting routine, smooth().

\begin{sphinxVerbatim}[commandchars=\\\{\}]
\PYG{k+kn}{import} \PYG{n+nn}{numpy} \PYG{k}{as} \PYG{n+nn}{np}
\PYG{k+kn}{from} \PYG{n+nn}{maxsmooth}\PYG{n+nn}{.}\PYG{n+nn}{DCF} \PYG{k+kn}{import} \PYG{n}{smooth}

\PYG{n}{x} \PYG{o}{=} \PYG{n}{np}\PYG{o}{.}\PYG{n}{load}\PYG{p}{(}\PYG{l+s+s1}{\PYGZsq{}}\PYG{l+s+s1}{Data/x.npy}\PYG{l+s+s1}{\PYGZsq{}}\PYG{p}{)}
\PYG{n}{y} \PYG{o}{=} \PYG{n}{np}\PYG{o}{.}\PYG{n}{load}\PYG{p}{(}\PYG{l+s+s1}{\PYGZsq{}}\PYG{l+s+s1}{Data/y.npy}\PYG{l+s+s1}{\PYGZsq{}}\PYG{p}{)}

\PYG{n}{N}\PYG{o}{=}\PYG{l+m+mi}{10}
\end{sphinxVerbatim}

There are several requirements needed to define a new basis function completely
for \sphinxcode{\sphinxupquote{maxsmooth}} to be able to fit it. They are as summarized below and then
examples of each are given in more detail,
\begin{itemize}
\item {} 
\sphinxstylestrong{args:} Additional non\sphinxhyphen{}standard  arguments needed in the definition of the
basis. The standard arguments are the data (x and y), the order of the fit N,
the pivot point about which a model can be fit,
the derivative order \({m}\) and the params. While the
pivot point is not strictly needed it is a required argument for the
functions defining a new basis to help the user in their definition.

\item {} 
\sphinxstylestrong{basis\_functions:} This function defines the basis of the DCF model,
\({\phi}\) where the model can be generally defined as,
\begin{equation*}
\begin{split}y = \sum_{k = 0}^N a_k \phi_k(x)\end{split}
\end{equation*}
where \({a_k}\) are the fit parameters.

\item {} 
\sphinxstylestrong{model:} This is the function described by the equation above.

\item {} 
\sphinxstylestrong{derivative:} This function defines the \({m^{th}}\) order derivative.

\item {} 
\sphinxstylestrong{derivative\_pre:} This function defines the prefactors,
\({\mathbf{G}}\) on the derivatives where \sphinxcode{\sphinxupquote{CVXOPT}}, the quadratic
programming routine used, evaluates the constraints as,
\begin{equation*}
\begin{split}\mathbf{Ga} \leq \mathbf{h}\end{split}
\end{equation*}
where \({\mathbf{a}}\) is the matrix of parameters and \({\mathbf{h}}\)
is the matrix of constraint limits. For more details on this see the \sphinxcode{\sphinxupquote{maxsmooth}}
paper.

\end{itemize}

We can begin defining our new basis function by defining the additional arguments
needed to fit the model as a list,

\begin{sphinxVerbatim}[commandchars=\\\{\}]
\PYG{n}{arguments} \PYG{o}{=} \PYG{p}{[}\PYG{n}{x}\PYG{p}{[}\PYG{o}{\PYGZhy{}}\PYG{l+m+mi}{1}\PYG{p}{]}\PYG{o}{*}\PYG{l+m+mi}{10}\PYG{p}{,} \PYG{n}{y}\PYG{p}{[}\PYG{o}{\PYGZhy{}}\PYG{l+m+mi}{1}\PYG{p}{]}\PYG{o}{*}\PYG{l+m+mi}{10}\PYG{p}{]}
\end{sphinxVerbatim}

The next step is to define the basis functions \({\phi}\). This needs to be
done in a function that has the arguments \sphinxstyleemphasis{(x, y, pivot\_point, N, *args)}. \textquotesingle{}args\textquotesingle{}
is optional but since we need them for this basis we are passing it in.

The basis functions, \({\phi}\), should be an array of dimensions len(x)
by N and consequently evaluated at each N and x data point as shown below.

\begin{sphinxVerbatim}[commandchars=\\\{\}]
\PYG{k}{def} \PYG{n+nf}{basis\PYGZus{}functions}\PYG{p}{(}\PYG{n}{x}\PYG{p}{,} \PYG{n}{y}\PYG{p}{,} \PYG{n}{pivot\PYGZus{}point}\PYG{p}{,} \PYG{n}{N}\PYG{p}{,} \PYG{o}{*}\PYG{n}{args}\PYG{p}{)}\PYG{p}{:}

    \PYG{n}{phi} \PYG{o}{=} \PYG{n}{np}\PYG{o}{.}\PYG{n}{empty}\PYG{p}{(}\PYG{p}{[}\PYG{n+nb}{len}\PYG{p}{(}\PYG{n}{x}\PYG{p}{)}\PYG{p}{,} \PYG{n}{N}\PYG{p}{]}\PYG{p}{)}
    \PYG{k}{for} \PYG{n}{h} \PYG{o+ow}{in} \PYG{n+nb}{range}\PYG{p}{(}\PYG{n+nb}{len}\PYG{p}{(}\PYG{n}{x}\PYG{p}{)}\PYG{p}{)}\PYG{p}{:}
        \PYG{k}{for} \PYG{n}{i} \PYG{o+ow}{in} \PYG{n+nb}{range}\PYG{p}{(}\PYG{n}{N}\PYG{p}{)}\PYG{p}{:}
            \PYG{n}{phi}\PYG{p}{[}\PYG{n}{h}\PYG{p}{,} \PYG{n}{i}\PYG{p}{]} \PYG{o}{=} \PYG{n}{args}\PYG{p}{[}\PYG{l+m+mi}{1}\PYG{p}{]}\PYG{o}{*}\PYG{p}{(}\PYG{n}{x}\PYG{p}{[}\PYG{n}{h}\PYG{p}{]}\PYG{o}{/}\PYG{n}{args}\PYG{p}{[}\PYG{l+m+mi}{0}\PYG{p}{]}\PYG{p}{)}\PYG{o}{*}\PYG{o}{*}\PYG{n}{i}

    \PYG{k}{return} \PYG{n}{phi}
\end{sphinxVerbatim}

We can define the model that we are fitting in a function like that shown below.
This is used for evaluating \({\chi^2}\) and returning the optimum fitted model
once the code has finished running. It requires the arguments
\sphinxstyleemphasis{(x, y, pivot\_point, N, params, *args)} in that order and again where \textquotesingle{}args\textquotesingle{} is optional.
\textquotesingle{}params\textquotesingle{} is the parameters of the fit, \({\mathbf{a}}\) which should have length
\({N}\).

The function should return the fitted estimate of y.

\begin{sphinxVerbatim}[commandchars=\\\{\}]
\PYG{k}{def} \PYG{n+nf}{model}\PYG{p}{(}\PYG{n}{x}\PYG{p}{,} \PYG{n}{y}\PYG{p}{,} \PYG{n}{pivot\PYGZus{}point}\PYG{p}{,} \PYG{n}{N}\PYG{p}{,} \PYG{n}{params}\PYG{p}{,} \PYG{o}{*}\PYG{n}{args}\PYG{p}{)}\PYG{p}{:}

    \PYG{n}{y\PYGZus{}sum} \PYG{o}{=} \PYG{n}{args}\PYG{p}{[}\PYG{l+m+mi}{1}\PYG{p}{]}\PYG{o}{*}\PYG{n}{np}\PYG{o}{.}\PYG{n}{sum}\PYG{p}{(}\PYG{p}{[}
        \PYG{n}{params}\PYG{p}{[}\PYG{n}{i}\PYG{p}{]}\PYG{o}{*}\PYG{p}{(}\PYG{n}{x}\PYG{o}{/}\PYG{n}{args}\PYG{p}{[}\PYG{l+m+mi}{0}\PYG{p}{]}\PYG{p}{)}\PYG{o}{*}\PYG{o}{*}\PYG{n}{i}
        \PYG{k}{for} \PYG{n}{i} \PYG{o+ow}{in} \PYG{n+nb}{range}\PYG{p}{(}\PYG{n}{N}\PYG{p}{)}\PYG{p}{]}\PYG{p}{,} \PYG{n}{axis}\PYG{o}{=}\PYG{l+m+mi}{0}\PYG{p}{)}

    \PYG{k}{return} \PYG{n}{y\PYGZus{}sum}
\end{sphinxVerbatim}

Next we have to define a function for the derivatives of the model which
takes arguments \sphinxstyleemphasis{(m, x, y, N, pivot\_point, params, *args)} where \({m}\) is
the derivative order. The function should return the \({m^{th}}\) order
derivative evaluation and is used for checking that the constraints have been
met and returning the derivatives of the optimum fit to the user.

\begin{sphinxVerbatim}[commandchars=\\\{\}]
\PYG{k}{def} \PYG{n+nf}{derivative}\PYG{p}{(}\PYG{n}{m}\PYG{p}{,} \PYG{n}{x}\PYG{p}{,} \PYG{n}{y}\PYG{p}{,} \PYG{n}{N}\PYG{p}{,} \PYG{n}{pivot\PYGZus{}point}\PYG{p}{,} \PYG{n}{params}\PYG{p}{,} \PYG{o}{*}\PYG{n}{args}\PYG{p}{)}\PYG{p}{:}

    \PYG{n}{mth\PYGZus{}order\PYGZus{}derivative} \PYG{o}{=} \PYG{p}{[}\PYG{p}{]}
    \PYG{k}{for} \PYG{n}{i} \PYG{o+ow}{in} \PYG{n+nb}{range}\PYG{p}{(}\PYG{n}{N}\PYG{p}{)}\PYG{p}{:}
        \PYG{k}{if} \PYG{n}{i} \PYG{o}{\PYGZlt{}}\PYG{o}{=} \PYG{n}{m} \PYG{o}{\PYGZhy{}} \PYG{l+m+mi}{1}\PYG{p}{:}
            \PYG{n}{mth\PYGZus{}order\PYGZus{}derivative}\PYG{o}{.}\PYG{n}{append}\PYG{p}{(}\PYG{p}{[}\PYG{l+m+mi}{0}\PYG{p}{]}\PYG{o}{*}\PYG{n+nb}{len}\PYG{p}{(}\PYG{n}{x}\PYG{p}{)}\PYG{p}{)}
    \PYG{k}{for} \PYG{n}{i} \PYG{o+ow}{in} \PYG{n+nb}{range}\PYG{p}{(}\PYG{n}{N} \PYG{o}{\PYGZhy{}} \PYG{n}{m}\PYG{p}{)}\PYG{p}{:}
            \PYG{n}{mth\PYGZus{}order\PYGZus{}derivative\PYGZus{}term} \PYG{o}{=} \PYG{n}{args}\PYG{p}{[}\PYG{l+m+mi}{1}\PYG{p}{]}\PYG{o}{*}\PYG{n}{np}\PYG{o}{.}\PYG{n}{math}\PYG{o}{.}\PYG{n}{factorial}\PYG{p}{(}\PYG{n}{m}\PYG{o}{+}\PYG{n}{i}\PYG{p}{)} \PYG{o}{/} \PYGZbs{}
                \PYG{n}{np}\PYG{o}{.}\PYG{n}{math}\PYG{o}{.}\PYG{n}{factorial}\PYG{p}{(}\PYG{n}{i}\PYG{p}{)} \PYG{o}{*} \PYGZbs{}
                \PYG{n}{params}\PYG{p}{[}\PYG{n+nb}{int}\PYG{p}{(}\PYG{n}{m}\PYG{p}{)}\PYG{o}{+}\PYG{n}{i}\PYG{p}{]}\PYG{o}{*}\PYG{p}{(}\PYG{n}{x}\PYG{p}{)}\PYG{o}{*}\PYG{o}{*}\PYG{n}{i} \PYG{o}{/} \PYGZbs{}
                \PYG{p}{(}\PYG{n}{args}\PYG{p}{[}\PYG{l+m+mi}{0}\PYG{p}{]}\PYG{p}{)}\PYG{o}{*}\PYG{o}{*}\PYG{p}{(}\PYG{n}{i} \PYG{o}{+} \PYG{l+m+mi}{1}\PYG{p}{)}
            \PYG{n}{mth\PYGZus{}order\PYGZus{}derivative}\PYG{o}{.}\PYG{n}{append}\PYG{p}{(}
                \PYG{n}{mth\PYGZus{}order\PYGZus{}derivative\PYGZus{}term}\PYG{p}{)}

    \PYG{k}{return} \PYG{n}{mth\PYGZus{}order\PYGZus{}derivative}
\end{sphinxVerbatim}

Finally we have to define \({\mathbf{G}}\) which is used by \sphinxcode{\sphinxupquote{CVXOPT}} to
build the derivatives and constrain the functions. It takes arguments
\sphinxstyleemphasis{(m, x, y, N, pivot\_point, *args)} and should return the prefactor on the
\({m^{th}}\) order derivative. For a more thorough definition of the
prefactor on the derivative and an explanation of how the problem is
constrained in quadratic programming see the \sphinxcode{\sphinxupquote{maxsmooth}} paper.

\begin{sphinxVerbatim}[commandchars=\\\{\}]
\PYG{k}{def} \PYG{n+nf}{derivative\PYGZus{}pre}\PYG{p}{(}\PYG{n}{m}\PYG{p}{,} \PYG{n}{x}\PYG{p}{,} \PYG{n}{y}\PYG{p}{,} \PYG{n}{N}\PYG{p}{,} \PYG{n}{pivot\PYGZus{}point}\PYG{p}{,} \PYG{o}{*}\PYG{n}{args}\PYG{p}{)}\PYG{p}{:}

    \PYG{n}{mth\PYGZus{}order\PYGZus{}derivative} \PYG{o}{=} \PYG{p}{[}\PYG{p}{]}
    \PYG{k}{for} \PYG{n}{i} \PYG{o+ow}{in} \PYG{n+nb}{range}\PYG{p}{(}\PYG{n}{N}\PYG{p}{)}\PYG{p}{:}
        \PYG{k}{if} \PYG{n}{i} \PYG{o}{\PYGZlt{}}\PYG{o}{=} \PYG{n}{m} \PYG{o}{\PYGZhy{}} \PYG{l+m+mi}{1}\PYG{p}{:}
            \PYG{n}{mth\PYGZus{}order\PYGZus{}derivative}\PYG{o}{.}\PYG{n}{append}\PYG{p}{(}\PYG{p}{[}\PYG{l+m+mi}{0}\PYG{p}{]}\PYG{o}{*}\PYG{n+nb}{len}\PYG{p}{(}\PYG{n}{x}\PYG{p}{)}\PYG{p}{)}
    \PYG{k}{for} \PYG{n}{i} \PYG{o+ow}{in} \PYG{n+nb}{range}\PYG{p}{(}\PYG{n}{N} \PYG{o}{\PYGZhy{}} \PYG{n}{m}\PYG{p}{)}\PYG{p}{:}
            \PYG{n}{mth\PYGZus{}order\PYGZus{}derivative\PYGZus{}term} \PYG{o}{=} \PYG{n}{args}\PYG{p}{[}\PYG{l+m+mi}{1}\PYG{p}{]}\PYG{o}{*}\PYG{n}{np}\PYG{o}{.}\PYG{n}{math}\PYG{o}{.}\PYG{n}{factorial}\PYG{p}{(}\PYG{n}{m}\PYG{o}{+}\PYG{n}{i}\PYG{p}{)} \PYG{o}{/} \PYGZbs{}
                \PYG{n}{np}\PYG{o}{.}\PYG{n}{math}\PYG{o}{.}\PYG{n}{factorial}\PYG{p}{(}\PYG{n}{i}\PYG{p}{)} \PYG{o}{*} \PYGZbs{}
                \PYG{p}{(}\PYG{n}{x}\PYG{p}{)}\PYG{o}{*}\PYG{o}{*}\PYG{n}{i} \PYG{o}{/} \PYGZbs{}
                \PYG{p}{(}\PYG{n}{args}\PYG{p}{[}\PYG{l+m+mi}{0}\PYG{p}{]}\PYG{p}{)}\PYG{o}{*}\PYG{o}{*}\PYG{p}{(}\PYG{n}{i} \PYG{o}{+} \PYG{l+m+mi}{1}\PYG{p}{)}
            \PYG{n}{mth\PYGZus{}order\PYGZus{}derivative}\PYG{o}{.}\PYG{n}{append}\PYG{p}{(}
                \PYG{n}{mth\PYGZus{}order\PYGZus{}derivative\PYGZus{}term}\PYG{p}{)}

    \PYG{k}{return} \PYG{n}{mth\PYGZus{}order\PYGZus{}derivative}
\end{sphinxVerbatim}

With our functions and additional arguments defined we can pass these
to the \sphinxcode{\sphinxupquote{maxsmooth}} smooth() function as is shown below. This overwrites the
built in DCF model but you are still able to modify the fit type i.e. testing all
available sign combinations or sampling them.

\begin{sphinxVerbatim}[commandchars=\\\{\}]
\PYG{n}{result} \PYG{o}{=} \PYG{n}{smooth}\PYG{p}{(}\PYG{n}{x}\PYG{p}{,} \PYG{n}{y}\PYG{p}{,} \PYG{n}{N}\PYG{p}{,}
    \PYG{n}{basis\PYGZus{}functions}\PYG{o}{=}\PYG{n}{basis\PYGZus{}functions}\PYG{p}{,} \PYG{n}{model}\PYG{o}{=}\PYG{n}{model}\PYG{p}{,}
    \PYG{n}{derivatives}\PYG{o}{=}\PYG{n}{derivative}\PYG{p}{,} \PYG{n}{der\PYGZus{}pres}\PYG{o}{=}\PYG{n}{derivative\PYGZus{}pre}\PYG{p}{,} \PYG{n}{args}\PYG{o}{=}\PYG{n}{arguments}\PYG{p}{)}
\end{sphinxVerbatim}

The output of the fit can be accessed as before,

\begin{sphinxVerbatim}[commandchars=\\\{\}]
\PYG{n+nb}{print}\PYG{p}{(}\PYG{l+s+s1}{\PYGZsq{}}\PYG{l+s+s1}{Objective Funtion Evaluations:}\PYG{l+s+se}{\PYGZbs{}n}\PYG{l+s+s1}{\PYGZsq{}}\PYG{p}{,} \PYG{n}{result}\PYG{o}{.}\PYG{n}{optimum\PYGZus{}chi}\PYG{p}{)}
\PYG{n+nb}{print}\PYG{p}{(}\PYG{l+s+s1}{\PYGZsq{}}\PYG{l+s+s1}{RMS:}\PYG{l+s+se}{\PYGZbs{}n}\PYG{l+s+s1}{\PYGZsq{}}\PYG{p}{,} \PYG{n}{result}\PYG{o}{.}\PYG{n}{rms}\PYG{p}{)}
\PYG{n+nb}{print}\PYG{p}{(}\PYG{l+s+s1}{\PYGZsq{}}\PYG{l+s+s1}{Parameters:}\PYG{l+s+se}{\PYGZbs{}n}\PYG{l+s+s1}{\PYGZsq{}}\PYG{p}{,} \PYG{n}{result}\PYG{o}{.}\PYG{n}{optimum\PYGZus{}params}\PYG{p}{)}
\PYG{n+nb}{print}\PYG{p}{(}\PYG{l+s+s1}{\PYGZsq{}}\PYG{l+s+s1}{Fitted y:}\PYG{l+s+se}{\PYGZbs{}n}\PYG{l+s+s1}{\PYGZsq{}}\PYG{p}{,} \PYG{n}{result}\PYG{o}{.}\PYG{n}{y\PYGZus{}fit}\PYG{p}{)}
\PYG{n+nb}{print}\PYG{p}{(}\PYG{l+s+s1}{\PYGZsq{}}\PYG{l+s+s1}{Sign Combinations:}\PYG{l+s+se}{\PYGZbs{}n}\PYG{l+s+s1}{\PYGZsq{}}\PYG{p}{,} \PYG{n}{result}\PYG{o}{.}\PYG{n}{optimum\PYGZus{}signs}\PYG{p}{)}
\PYG{n+nb}{print}\PYG{p}{(}\PYG{l+s+s1}{\PYGZsq{}}\PYG{l+s+s1}{Derivatives:}\PYG{l+s+se}{\PYGZbs{}n}\PYG{l+s+s1}{\PYGZsq{}}\PYG{p}{,} \PYG{n}{result}\PYG{o}{.}\PYG{n}{derivatives}\PYG{p}{)}
\end{sphinxVerbatim}


\section{Best Basis Example}
\label{\detokenize{source/maxsmooth:best-basis-example}}
This function can be used to identify which of the built in DCFs
fits the data best before running joint fits.

To use it we begin by loading in the data,

\begin{sphinxVerbatim}[commandchars=\\\{\}]
\PYG{k+kn}{import} \PYG{n+nn}{numpy} \PYG{k}{as} \PYG{n+nn}{np}

\PYG{n}{x} \PYG{o}{=} \PYG{n}{np}\PYG{o}{.}\PYG{n}{load}\PYG{p}{(}\PYG{l+s+s1}{\PYGZsq{}}\PYG{l+s+s1}{Data/x.npy}\PYG{l+s+s1}{\PYGZsq{}}\PYG{p}{)}
\PYG{n}{y} \PYG{o}{=} \PYG{n}{np}\PYG{o}{.}\PYG{n}{load}\PYG{p}{(}\PYG{l+s+s1}{\PYGZsq{}}\PYG{l+s+s1}{Data/y.npy}\PYG{l+s+s1}{\PYGZsq{}}\PYG{p}{)}
\end{sphinxVerbatim}

and then importing the basis\_test() function.

\begin{sphinxVerbatim}[commandchars=\\\{\}]
\PYG{k+kn}{from} \PYG{n+nn}{maxsmooth}\PYG{n+nn}{.}\PYG{n+nn}{best\PYGZus{}basis} \PYG{k+kn}{import} \PYG{n}{basis\PYGZus{}test}
\end{sphinxVerbatim}

To call the function we use,

\begin{sphinxVerbatim}[commandchars=\\\{\}]
\PYG{n}{basis\PYGZus{}test}\PYG{p}{(}\PYG{n}{x}\PYG{p}{,} \PYG{n}{y}\PYG{p}{,} \PYG{n}{base\PYGZus{}dir}\PYG{o}{=}\PYG{l+s+s1}{\PYGZsq{}}\PYG{l+s+s1}{examples/}\PYG{l+s+s1}{\PYGZsq{}}\PYG{p}{)}
\end{sphinxVerbatim}

The function only requires the data but we can provide it with a base directory,
fit type and range of DCF orders to test. By default it uses the sign sampling
algorithm and tests \({N = 3 - 13}\). The resultant graph is saved in the
base directory and the example generated here is shown below.

\noindent{\hspace*{\fill}\sphinxincludegraphics[width=400\sphinxpxdimen]{{Basis_functions}.png}\hspace*{\fill}}


\section{\protect\({\chi^2}\protect\) Distribution Example}
\label{\detokenize{source/maxsmooth:chi-2-distribution-example}}
This example will show you how to generate a plot of the \({\chi^2}\)
distribution as a function of the discrete sign combinations on the constrained
derivatives.

First you will need to import your data and fit this using \sphinxcode{\sphinxupquote{maxsmooth}} as
was done in the simple example code.

\begin{sphinxVerbatim}[commandchars=\\\{\}]
\PYG{k+kn}{import} \PYG{n+nn}{numpy} \PYG{k}{as} \PYG{n+nn}{np}

\PYG{n}{x} \PYG{o}{=} \PYG{n}{np}\PYG{o}{.}\PYG{n}{load}\PYG{p}{(}\PYG{l+s+s1}{\PYGZsq{}}\PYG{l+s+s1}{Data/x.npy}\PYG{l+s+s1}{\PYGZsq{}}\PYG{p}{)}
\PYG{n}{y} \PYG{o}{=} \PYG{n}{np}\PYG{o}{.}\PYG{n}{load}\PYG{p}{(}\PYG{l+s+s1}{\PYGZsq{}}\PYG{l+s+s1}{Data/y.npy}\PYG{l+s+s1}{\PYGZsq{}}\PYG{p}{)}

\PYG{k+kn}{from} \PYG{n+nn}{maxsmooth}\PYG{n+nn}{.}\PYG{n+nn}{DCF} \PYG{k+kn}{import} \PYG{n}{smooth}

\PYG{n}{N} \PYG{o}{=} \PYG{l+m+mi}{10}
\PYG{n}{result} \PYG{o}{=} \PYG{n}{smooth}\PYG{p}{(}\PYG{n}{x}\PYG{p}{,} \PYG{n}{y}\PYG{p}{,} \PYG{n}{N}\PYG{p}{,} \PYG{n}{base\PYGZus{}dir}\PYG{o}{=}\PYG{l+s+s1}{\PYGZsq{}}\PYG{l+s+s1}{examples/}\PYG{l+s+s1}{\PYGZsq{}}\PYG{p}{,}
  \PYG{n}{data\PYGZus{}save}\PYG{o}{=}\PYG{k+kc}{True}\PYG{p}{,} \PYG{n}{fit\PYGZus{}type}\PYG{o}{=}\PYG{l+s+s1}{\PYGZsq{}}\PYG{l+s+s1}{qp}\PYG{l+s+s1}{\PYGZsq{}}\PYG{p}{)}
\end{sphinxVerbatim}

Here we have used some additional keyword arguments for the \textquotesingle{}smooth\textquotesingle{} fitting
function. \textquotesingle{}data\_save\textquotesingle{} ensures that the files containing the tested sign combinations
and the corresponding objective function evaluations exist in the base directory
which we have changed to \textquotesingle{}base\_dir=\textquotesingle{}examples/\textquotesingle{}\textquotesingle{}. These files are essential for
the plotting the \({\chi^2}\) distribution and are not saved by \sphinxcode{\sphinxupquote{maxsmooth}}
without \textquotesingle{}data\_save=True\textquotesingle{}. We have also set the \textquotesingle{}fit\_type\textquotesingle{} to \textquotesingle{}qp\textquotesingle{} rather than the
default \textquotesingle{}qp\sphinxhyphen{}sign\_flipping\textquotesingle{}. This ensures that all of the available sign
combinations are tested rather than a sampled set giving us a full picture of the
distribution when we plot it. We have used the default DCF model to fit this data.

We can import the \textquotesingle{}chi\_plotter\textquotesingle{} like so,

\begin{sphinxVerbatim}[commandchars=\\\{\}]
\PYG{k+kn}{from} \PYG{n+nn}{maxsmooth}\PYG{n+nn}{.}\PYG{n+nn}{chidist\PYGZus{}plotter} \PYG{k+kn}{import} \PYG{n}{chi\PYGZus{}plotter}
\end{sphinxVerbatim}

and produce the fit which gets placed in the base directory with the following
code,

\begin{sphinxVerbatim}[commandchars=\\\{\}]
\PYG{n}{chi\PYGZus{}plotter}\PYG{p}{(}\PYG{n}{N}\PYG{p}{,} \PYG{n}{base\PYGZus{}dir}\PYG{o}{=}\PYG{l+s+s1}{\PYGZsq{}}\PYG{l+s+s1}{examples/}\PYG{l+s+s1}{\PYGZsq{}}\PYG{p}{,} \PYG{n}{fit\PYGZus{}type}\PYG{o}{=}\PYG{l+s+s1}{\PYGZsq{}}\PYG{l+s+s1}{qp}\PYG{l+s+s1}{\PYGZsq{}}\PYG{p}{)}
\end{sphinxVerbatim}

We pass the same \textquotesingle{}base\_dir\textquotesingle{} as before so that the plotter can find the correct output
files. We also give the function the same \textquotesingle{}fit\_type\textquotesingle{} used for the fitting which
ensures that the files can be read.

The resultant plot is shown below and the yellow star shows the global minimum.
This can be used to determine how well
the sign sampling approach using a descent and directional exploration
can find the global minimum. If the distribution looks like noise then it is
unlikely the sign sampling algorithm will consistently find the global minimum.
Rather it will likely repeatedly return the local minima found after the descent
algorithm and you should use the \textquotesingle{}qp\textquotesingle{} method testing all available sign combinations
in any future fits to the data with this DCF model.

\noindent{\hspace*{\fill}\sphinxincludegraphics[width=400\sphinxpxdimen]{{chi_distribution}.png}\hspace*{\fill}}


\section{Parameter Plotter Example}
\label{\detokenize{source/maxsmooth:parameter-plotter-example}}
We can assess the parameter space around the optimum solution
found using \sphinxcode{\sphinxupquote{maxsmooth}} with the param\_plotter() function.
This can help us identify how well a problem can be solved using the
sign sampling approach employed by \sphinxcode{\sphinxupquote{maxsmooth}} or simply
be used to identify correlations between the foreground parameters.
For more details on this see the \sphinxcode{\sphinxupquote{maxsmooth}} paper.

We begin by importing and fitting the data as with the chi\_plotter()
function illustrated above.

\begin{sphinxVerbatim}[commandchars=\\\{\}]
\PYG{k+kn}{import} \PYG{n+nn}{numpy} \PYG{k}{as} \PYG{n+nn}{np}

\PYG{n}{x} \PYG{o}{=} \PYG{n}{np}\PYG{o}{.}\PYG{n}{load}\PYG{p}{(}\PYG{l+s+s1}{\PYGZsq{}}\PYG{l+s+s1}{Data/x.npy}\PYG{l+s+s1}{\PYGZsq{}}\PYG{p}{)}
\PYG{n}{y} \PYG{o}{=} \PYG{n}{np}\PYG{o}{.}\PYG{n}{load}\PYG{p}{(}\PYG{l+s+s1}{\PYGZsq{}}\PYG{l+s+s1}{Data/y.npy}\PYG{l+s+s1}{\PYGZsq{}}\PYG{p}{)}

\PYG{k+kn}{from} \PYG{n+nn}{maxsmooth}\PYG{n+nn}{.}\PYG{n+nn}{DCF} \PYG{k+kn}{import} \PYG{n}{smooth}

\PYG{n}{N} \PYG{o}{=} \PYG{l+m+mi}{5}
\PYG{n}{result} \PYG{o}{=} \PYG{n}{smooth}\PYG{p}{(}\PYG{n}{x}\PYG{p}{,} \PYG{n}{y}\PYG{p}{,} \PYG{n}{N}\PYG{p}{,} \PYG{n}{base\PYGZus{}dir}\PYG{o}{=}\PYG{l+s+s1}{\PYGZsq{}}\PYG{l+s+s1}{examples/}\PYG{l+s+s1}{\PYGZsq{}}\PYG{p}{,} \PYG{n}{fit\PYGZus{}type}\PYG{o}{=}\PYG{l+s+s1}{\PYGZsq{}}\PYG{l+s+s1}{qp}\PYG{l+s+s1}{\PYGZsq{}}\PYG{p}{)}
\end{sphinxVerbatim}

We have changed the order of the fit to 5 to illustrate that for
order \({N \leq 5}\) and fits with derivatives \({m \geq 2}\) constrained
the function will plot each region of the graph corresponding to
different sign functions in a different colourmap. If the constraints are
different or the order is greater than 5 then the viable regions will have
a single colourmap. Invalid regions are plotted as black shaded colourmaps
and the contour lines are contours of \({\chi^2}\).

We can import the function like so,

\begin{sphinxVerbatim}[commandchars=\\\{\}]
\PYG{k+kn}{from} \PYG{n+nn}{maxsmooth}\PYG{n+nn}{.}\PYG{n+nn}{parameter\PYGZus{}plotter} \PYG{k+kn}{import} \PYG{n}{param\PYGZus{}plotter}
\end{sphinxVerbatim}

and access it using,

\begin{sphinxVerbatim}[commandchars=\\\{\}]
\PYG{n}{param\PYGZus{}plotter}\PYG{p}{(}\PYG{n}{result}\PYG{o}{.}\PYG{n}{optimum\PYGZus{}params}\PYG{p}{,} \PYG{n}{result}\PYG{o}{.}\PYG{n}{optimum\PYGZus{}signs}\PYG{p}{,}
    \PYG{n}{x}\PYG{p}{,} \PYG{n}{y}\PYG{p}{,} \PYG{n}{N}\PYG{p}{,} \PYG{n}{base\PYGZus{}dir}\PYG{o}{=}\PYG{l+s+s1}{\PYGZsq{}}\PYG{l+s+s1}{examples/}\PYG{l+s+s1}{\PYGZsq{}}\PYG{p}{)}
\end{sphinxVerbatim}

The function takes in the optimum parameters and signs found after the fit
as well as the data and order of the fit. There are a number of keyword arguments
detailed in the following section and the resultant fit is shown below. The
function by default samples the parameter ranges 50\% either side of the optimum
and calculates 50 samples for each parameter. In each panel the two
labelled parameters are varied while the others are maintained at their optimum
values.

\noindent\sphinxincludegraphics{{Parameter_plot}.png}


\chapter{Maxsmooth Functions}
\label{\detokenize{source/maxsmooth:maxsmooth-functions}}
This section details the specifics of the built in functions in \sphinxcode{\sphinxupquote{maxsmooth}} including
the relevant keyword arguments and default parameters for all. Where keyword arguments
are essential for the functions to run this is stated.


\section{smooth()}
\label{\detokenize{source/maxsmooth:module-maxsmooth.DCF}}\label{\detokenize{source/maxsmooth:smooth}}\index{module@\spxentry{module}!maxsmooth.DCF@\spxentry{maxsmooth.DCF}}\index{maxsmooth.DCF@\spxentry{maxsmooth.DCF}!module@\spxentry{module}}
\sphinxstyleemphasis{smooth}, as demonstrated in the examples section,
is used to call the fitting routine. There are a number
of \({^{**}}\) kwargs that can be assigned to the function which change how
the fit is performed, the model that is fit and various other attributes.
These are detailed below.
\index{smooth (class in maxsmooth.DCF)@\spxentry{smooth}\spxextra{class in maxsmooth.DCF}}

\begin{fulllineitems}
\phantomsection\label{\detokenize{source/maxsmooth:maxsmooth.DCF.smooth}}\pysiglinewithargsret{\sphinxbfcode{\sphinxupquote{class }}\sphinxcode{\sphinxupquote{maxsmooth.DCF.}}\sphinxbfcode{\sphinxupquote{smooth}}}{\emph{\DUrole{n}{x}}, \emph{\DUrole{n}{y}}, \emph{\DUrole{n}{N}}, \emph{\DUrole{o}{**}\DUrole{n}{kwargs}}}{}
\sphinxstylestrong{Parameters:}
\begin{quote}
\begin{description}
\item[{x: \sphinxstylestrong{numpy.array}}] \leavevmode
\begin{DUlineblock}{0em}
\item[] The x data points for the set being fitted.
\end{DUlineblock}

\item[{y: \sphinxstylestrong{numpy.array}}] \leavevmode
\begin{DUlineblock}{0em}
\item[] The y data points for fitting.
\end{DUlineblock}

\item[{N: \sphinxstylestrong{int}}] \leavevmode
\begin{DUlineblock}{0em}
\item[] The number of terms in the DCF.
\end{DUlineblock}

\end{description}
\end{quote}

\sphinxstylestrong{Kwargs:}
\begin{quote}
\begin{description}
\item[{fit\_type: \sphinxstylestrong{Default = \textquotesingle{}qp\sphinxhyphen{}sign\_flipping\textquotesingle{}}}] \leavevmode
\begin{DUlineblock}{0em}
\item[] This kwarg allows the user to
switch between sampling the available discrete sign spaces
(default) or testing all sign combinations on the derivatives
which can be accessed by setting to \textquotesingle{}qp\textquotesingle{}.
\end{DUlineblock}

\item[{model\_type: \sphinxstylestrong{Default = \textquotesingle{}difference\_polynomial\textquotesingle{}}}] \leavevmode
\begin{DUlineblock}{0em}
\item[] Allows the user to
access default Derivative Constrained Functions built into the
software. Available options include the default, \textquotesingle{}polynomial\textquotesingle{},
\textquotesingle{}normalised\_polynomial\textquotesingle{}, \textquotesingle{}legendre\textquotesingle{}, \textquotesingle{}log\_polynomial\textquotesingle{},
\textquotesingle{}loglog\_polynomial\textquotesingle{} and \textquotesingle{}exponential\textquotesingle{}. For more details on the
functional form of the built in basis see the \sphinxcode{\sphinxupquote{maxsmooth}}
paper.
\end{DUlineblock}

\end{description}

\sphinxstylestrong{pivot\_point: Default = len(x)//2 otherwise an integer between}
\sphinxstylestrong{\sphinxhyphen{}len(x) and len(x)}
\begin{quote}

\begin{DUlineblock}{0em}
\item[] Some of the built in
models rely on pivot points in the data sets which by defualt
is set as the middle index. This can be altered via
this kwarg which can occasionally lead to a better quality fit.
\end{DUlineblock}
\end{quote}
\begin{description}
\item[{base\_dir: \sphinxstylestrong{Default = \textquotesingle{}Fitted\_Output/\textquotesingle{}}}] \leavevmode
\begin{DUlineblock}{0em}
\item[] The location of the outputted
data from \sphinxcode{\sphinxupquote{maxsmooth}}. This must be a string and end in \textquotesingle{}/\textquotesingle{}.
If the file does not exist then \sphinxcode{\sphinxupquote{maxsmooth}} will create it.
By default the only outputted data is a summary of the best
fit but additional data can be recorded by setting the keyword
argument \textquotesingle{}data\_save = True\textquotesingle{}.
\end{DUlineblock}

\item[{data\_save: \sphinxstylestrong{Default = False}}] \leavevmode
\begin{DUlineblock}{0em}
\item[] By setting this to True the algorithm
will save every tested set of parameters, signs and objective
function evaluations into files in base\_dir. Theses files will
be over written on repeated runs but they are needed to run the
\textquotesingle{}chidist\_plotter\textquotesingle{}.
\end{DUlineblock}

\item[{all\_output: \sphinxstylestrong{Default = False}}] \leavevmode
\begin{DUlineblock}{0em}
\item[] If set to True this outputs to the
terminal every fit performed by the algorithm. By default the
only output is the optimal solution once the code is finished.
\end{DUlineblock}

\item[{cvxopt\_maxiter: \sphinxstylestrong{Default = 10000 else integer}}] \leavevmode
\begin{DUlineblock}{0em}
\item[] This shouldn\textquotesingle{}t need
changing for most problems however if \sphinxcode{\sphinxupquote{CVXOPT}} fails with a
\textquotesingle{}maxiters reached\textquotesingle{} error message this can be increased.
Doing so arbitrarily will however increase the run time of
\sphinxcode{\sphinxupquote{maxsmooth}}.
\end{DUlineblock}

\item[{initial\_params: \sphinxstylestrong{Default = None else list of length N}}] \leavevmode
\begin{DUlineblock}{0em}
\item[] Allows the user
to overwrite the default initial parameters used by \sphinxcode{\sphinxupquote{CVXOPT}}.
\end{DUlineblock}

\end{description}

\sphinxstylestrong{constraints: Default = 2 else an integer less than or equal}
\sphinxstylestrong{to N \sphinxhyphen{} 1}
\begin{quote}

\begin{DUlineblock}{0em}
\item[] The minimum constrained derivative order which is set by default
to 2 for a Maximally Smooth Function.
\end{DUlineblock}
\end{quote}
\begin{description}
\item[{zero\_crossings: \sphinxstylestrong{Default = None else list of integers}}] \leavevmode
\begin{DUlineblock}{0em}
\item[] Allows you to
specify if the conditions should be relaxed on any
of the derivatives between constraints and the highest order
derivative. e.g. a 6th order fit with just a constrained 2nd
and 3rd order derivative would have zero\_crossings = {[}4, 5{]}.
\end{DUlineblock}

\item[{cap: \sphinxstylestrong{Default = (len(available\_signs)//N) + N else an integer}}] \leavevmode
\begin{DUlineblock}{0em}
\item[] Determines the maximum number of signs explored either side of
the minimum \({\chi^2}\) value found after the decent
algorithm has terminated.
\end{DUlineblock}

\item[{chi\_squared\_limit: \sphinxstylestrong{Default = 2 else float or int}}] \leavevmode
\begin{DUlineblock}{0em}
\item[] The prefactor on the maximum allowed increase in \({\chi^2}\)
during the directional exploration which is defaulted at 2.
If this value multiplied by the minimum \({\chi^2}\)
value found after the descent algorithm is exceeded then the
exploration in one direction is stopped and started in the
other. For more details on this and \textquotesingle{}cap\textquotesingle{} see the \sphinxcode{\sphinxupquote{maxsmooth}}
paper.
\end{DUlineblock}

\end{description}

The following Kwargs can be used by the user to define their own basis
function and will overwrite the \textquotesingle{}model\_type\textquotesingle{} kwarg.

\sphinxstylestrong{basis\_function: Default = None else function with parameters}
\sphinxstylestrong{(x, y, pivot\_point, N)}
\begin{quote}

\begin{DUlineblock}{0em}
\item[] This is a function of basis functions
for the quadratic programming. The variable pivot\_point is the
index at the middle of the datasets x and y by default but can
be adjusted.
\end{DUlineblock}
\end{quote}

\sphinxstylestrong{model: Default = None else function with parameters}
\sphinxstylestrong{(x, y, pivot\_point, N, params)}
\begin{quote}

\begin{DUlineblock}{0em}
\item[] This is
a user defined function describing the model to be fitted to
the data.
\end{DUlineblock}
\end{quote}

\sphinxstylestrong{der\_pres: Default = None else function with parameters}
\sphinxstylestrong{(m, x, y, N, pivot\_point)}
\begin{quote}

\begin{DUlineblock}{0em}
\item[] This function describes the prefactors on the
mth order derivative used in defining the constraint.
\end{DUlineblock}
\end{quote}

\sphinxstylestrong{derivatives: Default = None else function with parameters}
\sphinxstylestrong{(m, x, y, N, pivot\_point, params)}
\begin{quote}

\begin{DUlineblock}{0em}
\item[] User defined function describing the mth
order derivative used to check that conditions are being met.
\end{DUlineblock}
\end{quote}
\begin{description}
\item[{\sphinxstylestrong{args: Default = None else list}}] \leavevmode
\begin{DUlineblock}{0em}
\item[] Extra arguments for \sphinxtitleref{smooth}
to pass to the functions detailed above.
\end{DUlineblock}

\end{description}
\end{quote}

\sphinxstylestrong{Output}
\begin{quote}
\begin{description}
\item[{.y\_fit: \sphinxstylestrong{numpy.array}}] \leavevmode
\begin{DUlineblock}{0em}
\item[] The fitted array of y data from smooth().
\end{DUlineblock}

\item[{.optimum\_chi: \sphinxstylestrong{float}}] \leavevmode
\begin{DUlineblock}{0em}
\item[] The optimum \({\chi^2}\) value for the fit calculated by,
\end{DUlineblock}
\begin{equation*}
\begin{split}{X^2=\sum(y-y_{fit})^2}.\end{split}
\end{equation*}
\item[{.optimum\_params: \sphinxstylestrong{numpy.array}}] \leavevmode
\begin{DUlineblock}{0em}
\item[] The set of parameters corresponding to the optimum fit.
\end{DUlineblock}

\item[{.rms: \sphinxstylestrong{float}}] \leavevmode
\begin{DUlineblock}{0em}
\item[] The rms value of the residuals \({y_{res}=y-y_{fit}}\)
calculated by,
\end{DUlineblock}
\begin{equation*}
\begin{split}{rms=\sqrt{\frac{\sum(y-y_{fit})^2}{n}}}\end{split}
\end{equation*}
where \(n\) is the number of data points.

\item[{.derivatives: \sphinxstylestrong{numpy.array}}] \leavevmode
\begin{DUlineblock}{0em}
\item[] The \(m^{th}\) order derivatives.
\end{DUlineblock}

\item[{.optimum\_signs: \sphinxstylestrong{numpy.array}}] \leavevmode
\begin{DUlineblock}{0em}
\item[] The sign combinations corresponding to the
optimal result. The nature of the constraint means that a
negative \sphinxcode{\sphinxupquote{maxsmooth}} sign implies a positive \({m^{th}}\)
order derivative and visa versa.
\end{DUlineblock}

\end{description}
\end{quote}

\end{fulllineitems}



\section{best\_basis()}
\label{\detokenize{source/maxsmooth:module-maxsmooth.best_basis}}\label{\detokenize{source/maxsmooth:best-basis}}\index{module@\spxentry{module}!maxsmooth.best\_basis@\spxentry{maxsmooth.best\_basis}}\index{maxsmooth.best\_basis@\spxentry{maxsmooth.best\_basis}!module@\spxentry{module}}
As demonstrated, this function allows you to test the built in basis and their
ability to
fit the data. It produces a plot that shows \({\chi^2}\) as a function of
\({N}\) for the 7 built in models and saves the figure to the base
directory.
\index{basis\_test (class in maxsmooth.best\_basis)@\spxentry{basis\_test}\spxextra{class in maxsmooth.best\_basis}}

\begin{fulllineitems}
\phantomsection\label{\detokenize{source/maxsmooth:maxsmooth.best_basis.basis_test}}\pysiglinewithargsret{\sphinxbfcode{\sphinxupquote{class }}\sphinxcode{\sphinxupquote{maxsmooth.best\_basis.}}\sphinxbfcode{\sphinxupquote{basis\_test}}}{\emph{\DUrole{n}{x}}, \emph{\DUrole{n}{y}}, \emph{\DUrole{o}{**}\DUrole{n}{kwargs}}}{}
\sphinxstylestrong{Parameters:}
\begin{quote}
\begin{description}
\item[{x: \sphinxstylestrong{numpy.array}}] \leavevmode
\begin{DUlineblock}{0em}
\item[] The x data points for the set being fitted.
\end{DUlineblock}

\item[{y: \sphinxstylestrong{numpy.array}}] \leavevmode
\begin{DUlineblock}{0em}
\item[] The y data points for fitting.
\end{DUlineblock}

\end{description}
\end{quote}

\sphinxstylestrong{Kwargs:}
\begin{quote}
\begin{description}
\item[{fit\_type: \sphinxstylestrong{Default = \textquotesingle{}qp\sphinxhyphen{}sign\_flipping\textquotesingle{}}}] \leavevmode
\begin{DUlineblock}{0em}
\item[] This kwarg allows the user to switch between sampling the
available discrete sign spaces (default)
or testing all sign combinations on the derivatives which can
be accessed by setting to \textquotesingle{}qp\textquotesingle{}.
\end{DUlineblock}

\item[{base\_dir: \sphinxstylestrong{Default = \textquotesingle{}Fitted\_Output/\textquotesingle{}}}] \leavevmode
\begin{DUlineblock}{0em}
\item[] The location of the outputted
graph from function. This must be a string and end in \textquotesingle{}/\textquotesingle{}. If
the file does not exist then the function will create it.
\end{DUlineblock}

\end{description}

\sphinxstylestrong{N: Default = {[}3, .., 13{]} in steps of 1 else list or numpy array}
\sphinxstylestrong{of integers}
\begin{quote}

\begin{DUlineblock}{0em}
\item[] The DCF orders to test each basis function with. In
some instances the basis function may fail for a given
\({N}\) and higher orders due to overflow/underflow
errors or \sphinxcode{\sphinxupquote{CVXOPT}} errors.
\end{DUlineblock}
\end{quote}
\end{quote}

\end{fulllineitems}



\section{chidist\_plotter()}
\label{\detokenize{source/maxsmooth:module-maxsmooth.chidist_plotter}}\label{\detokenize{source/maxsmooth:chidist-plotter}}\index{module@\spxentry{module}!maxsmooth.chidist\_plotter@\spxentry{maxsmooth.chidist\_plotter}}\index{maxsmooth.chidist\_plotter@\spxentry{maxsmooth.chidist\_plotter}!module@\spxentry{module}}
This function allows the user to produce plots of the \({\chi^2}\)
distribution as a function of the available discrete sign spaces for the
constrained derivatives. This can be used to identify whether or not the
problem is \sphinxtitleref{ill defined}, see the \sphinxcode{\sphinxupquote{maxsmooth}} paper for a definition,
and if it can be solved using the sign sampling approach.

It can also be used to determine whether or not the \textquotesingle{}cap\textquotesingle{} and maximum allowed
increase on the value of \({\chi^2}\) during the directional exploration
are sufficient to identify the global minimum for the problem.

The function is reliant on the output of the \sphinxcode{\sphinxupquote{maxsmooth}} smooth() function.
The required outputs can be saved when running smooth()
using the \textquotesingle{}data\_save = True\textquotesingle{} kwarg.
\index{chi\_plotter (class in maxsmooth.chidist\_plotter)@\spxentry{chi\_plotter}\spxextra{class in maxsmooth.chidist\_plotter}}

\begin{fulllineitems}
\phantomsection\label{\detokenize{source/maxsmooth:maxsmooth.chidist_plotter.chi_plotter}}\pysiglinewithargsret{\sphinxbfcode{\sphinxupquote{class }}\sphinxcode{\sphinxupquote{maxsmooth.chidist\_plotter.}}\sphinxbfcode{\sphinxupquote{chi\_plotter}}}{\emph{\DUrole{n}{N}}, \emph{\DUrole{o}{**}\DUrole{n}{kwargs}}}{}
\sphinxstylestrong{Parameters:}
\begin{quote}
\begin{description}
\item[{N: \sphinxstylestrong{int}}] \leavevmode
\begin{DUlineblock}{0em}
\item[] The number of terms in the DCF.
\end{DUlineblock}

\end{description}
\end{quote}

\sphinxstylestrong{Kwargs:}
\begin{quote}
\begin{description}
\item[{fit\_type: \sphinxstylestrong{Default = \textquotesingle{}qp\sphinxhyphen{}sign\_flipping\textquotesingle{}}}] \leavevmode
\begin{DUlineblock}{0em}
\item[] This kwarg is the same as for the smooth() function.
Here it allows the files to be read from the base
directory.
\end{DUlineblock}

\item[{base\_dir: \sphinxstylestrong{Default = \textquotesingle{}Fitted\_Output/\textquotesingle{}}}] \leavevmode
\begin{DUlineblock}{0em}
\item[] The location of the outputted
data from \sphinxcode{\sphinxupquote{maxsmooth}}. This must be a string and end in \textquotesingle{}/\textquotesingle{}
and must contain the files \textquotesingle{}Output\_Evaluations/\textquotesingle{} and
\textquotesingle{}Output\_Signs/\textquotesingle{} which can be obtained by running smooth() with
data\_save=True.
\end{DUlineblock}

\item[{chi: \sphinxstylestrong{Default = None else list or numpy array}}] \leavevmode
\begin{DUlineblock}{0em}
\item[] A list of
\({\chi^2}\) evaluations. If provided then this is used
over outputted data in the base directory. It must have the
same length as the ouputted signs in the file \textquotesingle{}Output\_Signs/\textquotesingle{}
in the base directory. It must also be ordered correctly
otherwise the returned graph will not be correct. A correct
ordering is one for which each entry in the array corresponds
to the correct sign combination in \textquotesingle{}Output\_Signs/\textquotesingle{}.
Typically this will not be needed but if the \({\chi^2}\)
evaluation in \textquotesingle{}Output\_Evaluations/\textquotesingle{} in the base directory
is not in the desired parameter space this can be useful.
For example the built in logarithmic model calculates
\({\chi^2}\) in logarithmic space. To plot the distribution
in linear space we can calculate
\({\chi^2}\) in linear space using a function for the model
and the tested parameters which are found in
\textquotesingle{}Output\_Parameters/\textquotesingle{} in the base directory.
\end{DUlineblock}

\end{description}

\sphinxstylestrong{constraints: Default = 2 else an integer less than or equal}
\sphinxstylestrong{to N \sphinxhyphen{} 1}
\begin{quote}

\begin{DUlineblock}{0em}
\item[] The minimum constrained derivative order which is set by default
to 2 for a Maximally Smooth Function. Used here to determine
the number of possible sign combinations available.
\end{DUlineblock}
\end{quote}
\begin{description}
\item[{zero\_crossings: \sphinxstylestrong{Default = None else list of integers}}] \leavevmode
\begin{DUlineblock}{0em}
\item[] Allows you to
specify if the conditions should be relaxed on any
of the derivatives between constraints and the highest order
derivative. e.g. a 6th order fit with just a constrained 2nd
and 3rd order derivative would have a zero\_crossings = {[}4, 5{]}.
Again this is used in determining the possible sign
combinations available.
\end{DUlineblock}

\item[{plot\_limits: \sphinxstylestrong{Default = False}}] \leavevmode
\begin{DUlineblock}{0em}
\item[] Determines whether the limits on
the directional exploration are plotted on top of the
\({\chi^2}\) distribution.
\end{DUlineblock}

\item[{cap: \sphinxstylestrong{Default = (len(available\_signs)//N) + N else an integer}}] \leavevmode
\begin{DUlineblock}{0em}
\item[] Determines the maximum number of signs explored either side of
the minimum \({\chi^2}\) value found after the
decent algorithm has terminated when running smooth(). Here
it is used when plot\_limits=True.
\end{DUlineblock}

\item[{chi\_squared\_limit: \sphinxstylestrong{Default = 2 else float or int}}] \leavevmode
\begin{DUlineblock}{0em}
\item[] The prefactor on the maximum allowed increase in \({\chi^2}\)
during the directional exploration which is defaulted at 2.
If this value multiplied by the minimum \({\chi^2}\)
value found after the descent algorithm is exceeded then the
exploration in one direction is stopped and started in the
other. For more details on this and \textquotesingle{}cap\textquotesingle{} see the \sphinxcode{\sphinxupquote{maxsmooth}}
paper. Again this is used here
when plot\_limits=True.
\end{DUlineblock}

\end{description}
\end{quote}

\end{fulllineitems}



\section{parameter\_plotter()}
\label{\detokenize{source/maxsmooth:module-maxsmooth.parameter_plotter}}\label{\detokenize{source/maxsmooth:parameter-plotter}}\index{module@\spxentry{module}!maxsmooth.parameter\_plotter@\spxentry{maxsmooth.parameter\_plotter}}\index{maxsmooth.parameter\_plotter@\spxentry{maxsmooth.parameter\_plotter}!module@\spxentry{module}}
This function allows you to plot the parameter space around the optimum
solution found when running \sphinxcode{\sphinxupquote{maxsmooth}} and visualise the constraints with
contour lines given by \({\chi^2}\).
\index{param\_plotter (class in maxsmooth.parameter\_plotter)@\spxentry{param\_plotter}\spxextra{class in maxsmooth.parameter\_plotter}}

\begin{fulllineitems}
\phantomsection\label{\detokenize{source/maxsmooth:maxsmooth.parameter_plotter.param_plotter}}\pysiglinewithargsret{\sphinxbfcode{\sphinxupquote{class }}\sphinxcode{\sphinxupquote{maxsmooth.parameter\_plotter.}}\sphinxbfcode{\sphinxupquote{param\_plotter}}}{\emph{\DUrole{n}{best\_params}}, \emph{\DUrole{n}{optimum\_signs}}, \emph{\DUrole{n}{x}}, \emph{\DUrole{n}{y}}, \emph{\DUrole{n}{N}}, \emph{\DUrole{o}{**}\DUrole{n}{kwargs}}}{}
\sphinxstylestrong{Parameters:}
\begin{quote}
\begin{description}
\item[{best\_params: \sphinxstylestrong{numpy.array}}] \leavevmode
\begin{DUlineblock}{0em}
\item[] The optimum parameters found when running
a DCF fit to the data.
\end{DUlineblock}

\item[{optimum\_signs: \sphinxstylestrong{numpy.array}}] \leavevmode
\begin{DUlineblock}{0em}
\item[] The optimum signs for the DCF fit which
are used when the derivatives are equal to 0 across the band.
\end{DUlineblock}

\item[{x: \sphinxstylestrong{numpy.array}}] \leavevmode
\begin{DUlineblock}{0em}
\item[] The x data points.
\end{DUlineblock}

\item[{y: \sphinxstylestrong{numpy.array}}] \leavevmode
\begin{DUlineblock}{0em}
\item[] The y data points.
\end{DUlineblock}

\item[{N: \sphinxstylestrong{int}}] \leavevmode
\begin{DUlineblock}{0em}
\item[] The number of terms in the DCF.
\end{DUlineblock}

\end{description}
\end{quote}

\sphinxstylestrong{Kwargs:}
\begin{quote}
\begin{description}
\item[{model\_type: \sphinxstylestrong{Default = \textquotesingle{}difference\_polynomial\textquotesingle{}}}] \leavevmode
\begin{DUlineblock}{0em}
\item[] The functional form of
the model being plotted. If a the user has defined their own
basis they can supply this with the Kwargs below and this
will be overwritten.
\end{DUlineblock}

\item[{base\_dir: \sphinxstylestrong{Default = \textquotesingle{}Fitted\_Output/\textquotesingle{}}}] \leavevmode
\begin{DUlineblock}{0em}
\item[] The location in which the
parameter plot is saved.
\end{DUlineblock}

\end{description}

\sphinxstylestrong{constraints: Default = 2 else an integer less than or equal}
\sphinxstylestrong{to N \sphinxhyphen{} 1}
\begin{quote}

\begin{DUlineblock}{0em}
\item[] The minimum constrained derivative order which is set by default
to 2 for a Maximally Smooth Function. Used here to
determine the number of possible sign combinations available.
\end{DUlineblock}
\end{quote}
\begin{description}
\item[{zero\_crossings: \sphinxstylestrong{Default = None else list of integers}}] \leavevmode
\begin{DUlineblock}{0em}
\item[] Allows you to
specify if the conditions should be relaxed on any
of the derivatives between constraints and the highest order
derivative. e.g. a 6th order fit with just a constrained
2nd and 3rd order derivative would have an
zero\_crossings = {[}4, 5{]}.
Again this is used in determining the possible sign
combinations available.
\end{DUlineblock}

\item[{samples: \sphinxstylestrong{Default = 50}}] \leavevmode
\begin{DUlineblock}{0em}
\item[] The sampling rate across the parameter ranges
defined with the optimum solution and width.
\end{DUlineblock}

\item[{width: \sphinxstylestrong{Default = 0.5}}] \leavevmode
\begin{DUlineblock}{0em}
\item[] The range of each parameter to explore. The
default value of 0.5 means that the \({\chi^2}\)
values for parameters ranging 50\% either side of the optimum
result are tested.
\end{DUlineblock}

\item[{warnings: \sphinxstylestrong{Default = True}}] \leavevmode
\begin{DUlineblock}{0em}
\item[] Used to highlight when a derivative is
0 across the band and that in these instances the optimum
signs are assumed for the colourmap if \({N \leq 5}\),
constraints=2 and the zero\_crossings is empty.
\end{DUlineblock}

\item[{girdlines: \sphinxstylestrong{Default = False}}] \leavevmode
\begin{DUlineblock}{0em}
\item[] Plots gridlines showing the central value
for each parameter in each panel of the plot.
\end{DUlineblock}

\end{description}

The following Kwargs are used to plot the parameter space for a user
defined basis function and will overwrite the \textquotesingle{}model\_type\textquotesingle{} kwarg.

\sphinxstylestrong{basis\_function: Default = None else function with parameters}
\sphinxstylestrong{(x, y, pivot\_point, N)}
\begin{quote}

\begin{DUlineblock}{0em}
\item[] This is a function of basis functions
for the quadratic programming. The variable pivot\_point is the
index at the middle of the datasets x and y by default but can
be adjusted.
\end{DUlineblock}
\end{quote}

\sphinxstylestrong{model: Default = None else function with parameters}
\sphinxstylestrong{(x, y, pivot\_point, N, params)}
\begin{quote}

\begin{DUlineblock}{0em}
\item[] This is
a user defined function describing the model to be fitted to
the data.
\end{DUlineblock}
\end{quote}

\sphinxstylestrong{der\_pres: Default = None else function with parameters}
\sphinxstylestrong{(m, x, y, N, pivot\_point)}
\begin{quote}

\begin{DUlineblock}{0em}
\item[] This function describes the prefactors on the
mth order derivative used in defining the constraint.
\end{DUlineblock}
\end{quote}

\sphinxstylestrong{derivatives: Default = None else function with parameters}
\sphinxstylestrong{(m, x, y, N, pivot\_point, params)}
\begin{quote}

\begin{DUlineblock}{0em}
\item[] User defined function describing the mth
order derivative used to check that conditions are being met.
\end{DUlineblock}
\end{quote}
\begin{description}
\item[{\sphinxstylestrong{args: Default = None else list}}] \leavevmode
\begin{DUlineblock}{0em}
\item[] Extra arguments for \sphinxtitleref{smooth}
to pass to the functions detailed above.
\end{DUlineblock}

\end{description}
\end{quote}

\end{fulllineitems}



\renewcommand{\indexname}{Python Module Index}
\begin{sphinxtheindex}
\let\bigletter\sphinxstyleindexlettergroup
\bigletter{m}
\item\relax\sphinxstyleindexentry{maxsmooth.best\_basis}\sphinxstyleindexpageref{source/maxsmooth:\detokenize{module-maxsmooth.best_basis}}
\item\relax\sphinxstyleindexentry{maxsmooth.chidist\_plotter}\sphinxstyleindexpageref{source/maxsmooth:\detokenize{module-maxsmooth.chidist_plotter}}
\item\relax\sphinxstyleindexentry{maxsmooth.DCF}\sphinxstyleindexpageref{source/maxsmooth:\detokenize{module-maxsmooth.DCF}}
\item\relax\sphinxstyleindexentry{maxsmooth.parameter\_plotter}\sphinxstyleindexpageref{source/maxsmooth:\detokenize{module-maxsmooth.parameter_plotter}}
\end{sphinxtheindex}

\renewcommand{\indexname}{Index}
\printindex
\end{document}